% !TEX TS-program = pdflatex
% !TEX encoding = UTF-8 Unicode

\documentclass[a4paper, titlepage=false, parskip=full-, 10pt]{scrartcl}

\usepackage[utf8]{inputenc}
\usepackage[T1]{fontenc}
\usepackage[english, ngerman]{babel}
\usepackage{babelbib}
\usepackage{hyperref}
\usepackage{listings}
\usepackage{framed}
\usepackage{color}
\usepackage{graphicx}
\usepackage[normalem]{ulem}
\usepackage{cancel}
\usepackage{amsmath}
\usepackage{amssymb}
\usepackage{amsthm}
\usepackage{algorithm}
\usepackage{algorithmic}
\usepackage{geometry}
\usepackage{subfigure}
\geometry{a4paper, top=20mm, left=35mm, right=25mm, bottom=40mm}

\newcounter{tasknbr}
\setcounter{tasknbr}{1}
\newenvironment{task}[1]{{\bf Aufgabe \arabic {tasknbr}\stepcounter{tasknbr}} (#1):\begin{enumerate}}{\end{enumerate}}
\newcommand{\subtask}[1]{\item[#1)]}

% Listings -----------------------------------------------------------------------------
\definecolor{red}{rgb}{.8,.1,.2}
\definecolor{blue}{rgb}{.2,.3,.7}
\definecolor{lightyellow}{rgb}{1.,1.,.97}
\definecolor{gray}{rgb}{.7,.7,.7}
\definecolor{darkgreen}{rgb}{0,.5,.1}
\definecolor{darkyellow}{rgb}{1.,.7,.3}
\lstloadlanguages{C++,[Objective]C,Java}
\lstset{
escapeinside={§§}{§§},
basicstyle=\ttfamily\footnotesize\mdseries,
columns=fullflexible,
keywordstyle=\bfseries\color{blue},
commentstyle=\color{darkgreen},      
stringstyle=\color{red},
numbers=left,
numberstyle=\ttfamily\scriptsize\color{gray},
breaklines=true,
showstringspaces=false,
tabsize=4,
captionpos=b,
float=htb,
frame=tb,
frameshape={RYR}{y}{y}{RYR},
rulecolor=\color{black},
xleftmargin=15pt,
xrightmargin=4pt,
aboveskip=\bigskipamount,
belowskip=\bigskipamount,
backgroundcolor=\color{lightyellow},
extendedchars=true,
belowcaptionskip=15pt}

%% Enter current values here: %%
\newcommand{\lecture}{Computer Vision WS15/16}
\newcommand{\tutor}{}
\newcommand{\assignmentnbr}{7}
\newcommand{\students}{Julius Auer}
%%-------------------------------------%%

\begin{document}  
{\small \textsl{\lecture \hfill \tutor}}
\hrule
\begin{center}
\textbf{Übungsblatt \assignmentnbr}\\
[\bigskipamount]
{\small \students}
\end{center}
\hrule

\begin{task}{Scale Space}
\item[]\emph{\begin{itemize}
\item Lies das Paper von David Lowe (2004)
\item Implementiere den Scale Space und die Berechnung der DoG-Pyramide (benutze die Standardparameter, wie im Paper)
\item Stelle die unterste und oberste DoG-Schicht für das Eingabebild Lenna.png dar (Abgabe 1.1)
\end{itemize}}

\end{task}

\begin{task}{Keypoint-Detektion und Aussieben}
\item[]\emph{\begin{itemize}
\item Implementiere die Kandidatensuche
\item Implementiere die Aussortierung von Keypoints mit schwachem Kontrast und geringer Krümmung (die genaue Lokalisierung sparen wir uns hier)
\item Plotte die übrigen Keypoints auf dem Eingabebild als Kreise, deren Radii proportional zum scale sind (Abgabe 2.1)
\end{itemize}}

\end{task}

\begin{task}{Keypoint-Orientierung}
\item[]\emph{\begin{itemize}
\item Implementiere die Bestimmung der Keypoint-Orientierung (ohne Parabelfit)
\item Plotte die Keypoints aus 2. auf dem Eingabebild als Kreise, deren Radii proportional zum scale sind und zusätzlich einen Pfeil besitzen, der die Richtung des Keypoints widerspiegelt. (Abgabe 3.1)
\end{itemize}}

\end{task}

\begin{task}{Keypoint-Matching}
\item[]\emph{\begin{itemize}
\item Implementiere eine Funktion, die, gegeben zwei Keypoint-Mengen, korrespondierende Keypoints identifiziert.
\item Implementiere eine Funktion, die diese Korrespondenzen bildlich darstellt (z.B. als Linien, die die Keypoints in den beiden Bildern verbinden)
\item In den Dateien locs\_X.csv und desc\_X.csv sind Keypoints und Deskriptoren für die Bilder Lenna.png und Lenna\_transformed.png gegeben. Stelle die Korrespondenzen grafisch dar. (Abgabe 4.1)
\end{itemize}}

\end{task}
\end{document}